%==============================================================================
% Example hogent-article: paper in English
%==============================================================================

\documentclass[dutch]{hogent-article}

\usepackage{parskip}

\setlength{\parskip}{0pt} % Pas deze waarde aan zoals gewenst
% Specify the BibLaTeX file(s) that contain(s) your bibliography
\addbibresource{paper.bib}

% Course information
\studyprogramme{Paper Hoe combineer je PRINCE2 of PMBOK met agile\linebreak werken? }
\course{Business Analysis}
\assignmenttype{Paper}
\academicyear{2023-2024}
\onecolumn

% Title of the paper
\title{Hoe combineer je PRINCE2 of PMBOK met agile\linebreak werken?}

% Authors and their emails. Each should be specified separately
\author{Tibo De Wolf}
\email{tibo.dewolf@student.hogent.be}

\author{Dorus Braems}
\email{dorus.braems@student.hogent.be}

\author{Arthur Neirynck}
\email{arthur.neirynck@student.hogent.be}

\author{Sam Michiels}
\email{sam.michiels@student.hogent.be}


\newcommand{\klas}[1]{\def\@klas{#1}}
\newcommand{\groepsnummer}[1]{\def\@groepsnummer{#1}}
\newcommand{\printklas}{\@klas}
\newcommand{\printgroepsnummer}{\@groepsnummer}

\klas{G3C1}
\groepsnummer{SL_CD_06}
% Supervisor, i.e. the lecturer that will read and grade this paper
%  You can specify the role of the supervisor with an optional argument, e.g.
%   \supervisor[Advisor]{NAME}
%   \supervisor[Lecturer]{NAME}
%  Remark that there is only room for one supervisor

% Link to a Github repo where the project code is kept

% Specify the specialisation within the study programme. Choose from this
% list:
%
% - Mobile \& Enterprise development
% - AI \& Data Engineering
% - Functional \& Business Analysis
% - System \& Network Administrator
% - Mainframe Expert
% - If the subject doesn't fit, specify another (or leave the command out
%   entirely)
\specialisation{System \& Network Administrator}

% Specify some keywords that describe the paper subject
\keywords{Projectmanagement, Systeembeheer}

% Optionally, set the title/link colors to one of HOGENT's corporate brand
% colors (default for both is hogent-blue)
% See hogent-article.cls to check which colors are supported
\colorlet{title}{hogent-darkgreen}
\colorlet{links}{hogent-grey}

\begin{document}

\begin{abstract}

\end{abstract}
\bigskip
\textbf{Klas:} G2C1
\newline
\textbf{Groepnummer:} 


\newpage

\tableofcontents
\pagebreak

\section{Introductie}%
\label{sec:introductie}

\section{Literatuurstudie}%
\label{sec:literatuurstudie}


\section{Interviews}
\label{sec:interviews}


\subsection{Gestelde vragen}%
\label{ssec:gestelde-vragen}

Tijdens deze interviews hebben we de volgende vragen gesteld:

\subsubsection{Huidig Project Management Framework}
\label{ssec:Huidig Project Management Framework}

\begin{itemize}
    \item Hoe zou je het huidige project management framework binnen het bedrijf beschrijven? Maakt het gebruik van PRINCE2/PMBOK, agile, of een combinatie daarvan?
\end{itemize}

\subsubsection{Mogelijkheden van een Hybride PRINCE2/PMBOK en Agile Aanpak}
\label{ssec:Mogelijkheden en uitdagingen van een Hybride PRINCE2/PMBOK en Agile Aanpak}

\begin{itemize}
    \item Hoewel Agile flexibeler is, kan het een aantal  problemen met zich meebrengen zoals: scope creep, onrealistische verwachtingen, gebrek aan samenwerking en gebrek aan communicatie. Denkt u dat een combinatie van Prince2/PMBOK en agile deze problemen kan verhelpen?
\end{itemize}

\subsubsection{Integratie PRINCE2 Agile - Balanceren tussen Productgerichtheid en Flexibiliteit}
\label{ssec:Integratie PRINCE2 Agile - Balanceren tussen Productgerichtheid en Flexibiliteit}

\begin{itemize}
    \item PRINCE2 helpt om te bepalen welke producten nodig zijn om aan de business case te voldoen, terwijl Agile-projecten zich meestal richten op het maken van één product en dat goed doen. Hoe brengt PRINCE2 agile deze 2 perspectieven samen en zorgt het ervoor dat er geen essentiële producten over het hoofd worden gezien, terwijl het tegelijkertijd de flexibiliteit van Agile benut om de benodigde producten te creëren?
\end{itemize}

\subsubsection{Uitdagingen bij Integratie van PRINCE2 en Agile in de Organisatie}
\label{ssec:Uitdagingen en Overwinningen bij Integratie van PRINCE2 en Agile in de Organisatie}

\begin{itemize}
    \item Het mengen van Agile met PRINCE2 zou in eerste instantie gezien kunnen worden als het toevoegen van zware processen aan Agile. Zijn er uitdagingen of scepticisme geweest van PRINCE2 of Agile beoefenaars binnen uw organisatie, en hoe hebt u deze zorgen aangepakt?
\end{itemize}

\subsubsection{Uitdagingen Agile met PRINCE2 - Budget en Timing}
\label{ssec:Uitdagingen Agile met PRINCE2 - Budget en Timing}

\begin{itemize}
    \item Wat zijn de uitdagingen of problemen die Agile met PRINCE2 ervaart in verband met budget en timing?
\end{itemize}

\subsubsection{Tools voor Integratie PRINCE2/PMBOK en Agile}
\label{ssec:geslaagde-en-niet-geslaagde-projecten}

\begin{itemize}
    \item Welke tools worden zoal gebruikt om PRINCE2/PMBOK en agile met elkaar te integreren ?
\end{itemize}

\subsection{Cegeka}
\label{ssec:Inetum-Cegeka}



\subsection{Inetum-realdolmen}
\label{ssec:interview-Inetum-realdolmen}



\section{Theorie vs praktijk}
\label{sec:theorie-vs-pratijk-sfpd}



\section{Conclusie}%
\label{sec:conclusie}


\section{Reflectie}
\label{sec:reflectie}



%\printbibliography[heading=bibintoc]

\end{document}
