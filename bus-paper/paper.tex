%==============================================================================
% Example hogent-article: paper in English
%==============================================================================

\documentclass[dutch]{hogent-article}

\usepackage{parskip}

\setlength{\parskip}{0pt} % Pas deze waarde aan zoals gewenst
% Specify the BibLaTeX file(s) that contain(s) your bibliography
\addbibresource{paper.bib}

% Course information
\studyprogramme{Paper Hoe combineer je PRINCE2 of PMBOK met agile\linebreak werken? }
\course{Business Analysis}
\assignmenttype{Paper}
\academicyear{2023-2024}
\onecolumn

% Title of the paper
\title{Hoe combineer je PRINCE2 of PMBOK met agile\linebreak werken?}

% Authors and their emails. Each should be specified separately
\author{Tibo De Wolf}
\email{tibo.dewolf@student.hogent.be}

\author{Dorus Braems}
\email{dorus.braems@student.hogent.be}

\author{Arthur Neirynck}
\email{arthur.neirynck@student.hogent.be}

\author{Sam Michiels}
\email{sam.michiels@student.hogent.be}


\newcommand{\klas}[1]{\def\@klas{#1}}
\newcommand{\groepsnummer}[1]{\def\@groepsnummer{#1}}
\newcommand{\printklas}{\@klas}
\newcommand{\printgroepsnummer}{\@groepsnummer}

\klas{G3C1}
\groepsnummer{SL_CD_06}
% Supervisor, i.e. the lecturer that will read and grade this paper
%  You can specify the role of the supervisor with an optional argument, e.g.
%   \supervisor[Advisor]{NAME}
%   \supervisor[Lecturer]{NAME}
%  Remark that there is only room for one supervisor

% Link to a Github repo where the project code is kept

% Specify the specialisation within the study programme. Choose from this
% list:
%
% - Mobile \& Enterprise development
% - AI \& Data Engineering
% - Functional \& Business Analysis
% - System \& Network Administrator
% - Mainframe Expert
% - If the subject doesn't fit, specify another (or leave the command out
%   entirely)
\specialisation{System \& Network Administrator}

% Specify some keywords that describe the paper subject
\keywords{Projectmanagement, Systeembeheer}

% Optionally, set the title/link colors to one of HOGENT's corporate brand
% colors (default for both is hogent-blue)
% See hogent-article.cls to check which colors are supported
\colorlet{title}{hogent-darkgreen}
\colorlet{links}{hogent-grey}

\begin{document}

\begin{abstract}

\end{abstract}
\bigskip
\textbf{Klas:} G2C1
\newline
\textbf{Groepnummer:} 


\newpage

\tableofcontents
\pagebreak

\section{Introductie}%
\label{sec:introductie}

Merendeel van de hedendaagse projectmanagment strategie\"en leggen vaak de focus op softwareontwikkeling. Echter, in de wereld van systeembeheer rijst de vraag: kunnen deze gevestigde projectmanagementmethoden naadloos worden toegepast op de uitdagende context van systeembeheerprojecten, zoals bijvoorbeeld de migratie naar een nieuwe versie van een besturingssysteem?
\newline

Met dit onderzoek willen we focus op softwareontwikkeling doorbreken, en trachten we de gelijkenissen en verschillen te ontdekken tussen projecten in systeembeheer en die in softwareontwikkeling. Zijn de fundamenten van projectmanagement overdraagbaar, of vereisen systeembeheerprojecten een unieke benadering? Deze paper verkent deze vragen om een dieper inzicht te verschaffen in de projectmanagementdynamiek binnen het domein van systeembeheer.


\section{Literatuurstudie}%
\label{sec:literatuurstudie}


\section{Interviews}
\label{sec:interviews}


\subsection{Gestelde vragen}%
\label{ssec:gestelde-vragen}

Tijdens deze interviews hebben we de volgende vragen gesteld:

\subsubsection{Toepassingen projectmethodologieën}
\label{ssec:toepassingen-projectmethodologieën}

\begin{itemize}
    \item Kun je wat meer in detail treden over de specifieke projectmanagementmethodologieën die vaak worden toegepast binnen IT-projecten en specifiek binnen systeembeheerprojecten bij [naam bedrijf]
    \item Is dat eerder op een Scrum gewijze manier, KanBan of zelfs een Waterfall methode, of wordt het eerder bepaald door de aarde van het project, of zelfs a.d.h.v. de voorkeur van het team?
\end{itemize}

\subsubsection{Vergelijking met Softwareontwikkelingsprojecten}
\label{ssec:vergelijking-met-softwareontwikkelingsprojecten}

\begin{itemize}
    \item We weten allemaal dat meeste van deze methodologieën gericht zijn op Softwareontwikkelingsprojecten, maar wat zijn nu de overeenkomsten tussen deze projecten en een typisch systeembeheerproject?
    \item Kan men volgens u, bijv. sprints toepassen op een systeembeheerproject i.p.v. te werken met een ticketingsystem?
    \item Zo ja, wat zijn dan specifieke aanpassingen die gemaakt moeten worden om dit te doen werken?
\end{itemize}

\subsubsection{Belangrijkste Uitdagingen}
\label{ssec:belangrijkste-uitdagingen}

\begin{itemize}
    \item Welke uitdagingen komen vaak voor bij systeembeheerprojecten bij [bedrijf], en hoe verschillen deze van uitdagingen binnen andere IT-projecten?
    \item Is het dan de rol van de projectleider om deze op te lossen?
\end{itemize}

\subsubsection{Aanpak bij Veranderingen}
\label{ssec:aanpak-bij-veranderingen}

\begin{itemize}
    \item Hoe past [bedrijf] de projectmanagementbenadering aan wanneer er onverwachte wijzigingen optreden, bijvoorbeeld de noodzaak om over te schakelen naar een ander besturingssysteem als reactie op externe gebeurtenissen, zoals een project dat End-of-Life is?
\end{itemize}

\subsubsection{Evaluatie van Projecten}
\label{ssec:evaluatie-van-projecten}

\begin{itemize}
    \item Hoe evalueert [bedrijf] het succes van systeembeheerprojecten? Zijn er specifieke KPI's of evaluatiemethoden die worden toegepast?
\end{itemize}

\subsubsection{Geslaagde en Niet Geslaagde Projecten}
\label{ssec:geslaagde-en-niet-geslaagde-projecten}

\begin{itemize}
    \item Zijn projecten die tot een goed eind gebracht werden, dat jullie met ons kunnen delen?
    \item Zijn er ook projecten die niet geslaagd zijn, lag dit dan aan de gekozen methodologië, of andere zaken?
    \item Hebben jullie hier dan bepaalde lessen uit geleerd, en hoe ging het team hier dan mee om?
\end{itemize}

\subsection{Federale Pensioendienst}
\label{ssec:interview-fpd}

Bij infrastructuurteam van de Federale Pensioendienst werkt er vaak \'e\'en persoon aan het hele project, aangezien deze vaak klein genoeg zijn. Deze persoon is verantwoordelijk voor zowel de analyse, effectieve uitwerking en tot slot het testen. De manier waarop het project dan uiteindelijk wordt uitgewerkt wordt bepaald door het zogenaamde Project Management Office (PMO).
\newline
    Indien ze moeten samen werken met andere teams merken ze vaak dat de communicatie stroef verloopt tussen beide partijen. Echter herkenen ze dit volop, en proberen ze hier dan ook aan te werken.
\newline

Indien een bepaald project waar ze afhankelijk van zijn end of life wordt verklaard worden er afspraken gemaakt over welke aanpassingen er zullen gebeuren. Er wordt een simultane overgang voorzien waarbij zowel het oude als het nieuwe systeem worden ondersteund, maar geleidelijk wordt er volledig overgestapt op het nieuwe systeem.
\newline

Als we vroegen of er bepaalde KPI's zijn die voor elk project worden opgesteld, gaven ze aan dat ze dit wel hebben gedefinieerd, maar dat ze vaak geen tijd meer hebben om deze allemaal af te toetsen. Dit staat in contrast met de algemene

\subsection{KBC Bank \& Verzekering}
\label{ssec:interview-kbc}

In de loop der tijd hebben we gezien dat veel projectmanagementmethodologieën zijn opgezet met software-ontwikkeling in gedachten. De vraag die rijst is of deze methoden ook van toepassing zijn op systeembeheerprojecten, bijvoorbeeld bij een migratie naar een nieuwe versie van een besturingssysteem. We spraken met twee mensen van KBC om hun inzichten te verkrijgen.
\newline

Een interessante ontwikkeling die ze doormaakten, was het gebruik van een eigen systeem genaamd PLC (Project Life Cycle). In het begin was dit vrij gelijkaardig aan Prince2, maar na verloop van tijd werden er Agile principes aan toegevoegd. Op dit moment combineren ze beide methodologieën, maar het doel is om over te schakelen naar Scaled Agile. Het proces omvat veel inspanningen, wat ze zelf beschrijven als een "never-ending story."
\newline

Per afdeling is er een backlog met items die ze willen realiseren. Gedurende het jaar doorlopen ze vier cycli van elk drie maanden. In elke cyclus selecteren ze items uit de backlog en streven ernaar deze te realiseren, een aanpak vergelijkbaar met klassiek PI (Project Increment) planning.
\newline

Het team bevat een product owner en scrum master, en ze maken gebruik van een combinatie van Kanban en Scrum, genaamd ScrumBan. Over het algemeen is de implementatie Scrum-gebaseerd, terwijl het werk zelf meer neigt naar Kanban. Gedurende dit hele proces staat Jira aan hun zijde als ondersteunende tool.
\newline

Ze erkennen dat Agile meer gericht is op softwareontwikkeling en niet altijd naadloos overgaat naar infrastructuurprojecten. De diversiteit aan vaardigheden in dergelijke projecten, zoals databases en netwerkbeheer, vereist een aanpak die meer naar ScrumBan neigt om het team effectief samen te brengen.
\newline

Interessant genoeg hebben ze geen traditionele projectleiders meer. Ze hebben volledig de principes van SAFe Agile geïmplementeerd, wat betekent dat ze genoeg bijsturingsmomenten hebben. Na elke drie maanden is er een overleg om te bepalen of ze doorgaan met het project of het stoppen, en binnen die drie maanden zijn er iteraties of sprints met bijsturingsmogelijkheden.
\newline

Hoewel ze theoretisch KPI's hebben, merken ze dat, zoals zij het verwoorden, op een sneltrein zitten, waardoor KPI's uit het oog kunnen worden verloren. De scrum master speelt een cruciale rol in het monitoren en bijsturen. Evaluaties vinden niet alleen plaats aan het einde van het project, maar ook elke drie maanden, waarbij het team kijkt naar successen, uitdagingen en ruimte voor verbetering.
\newline

Belangrijk is ook de nadruk op retrospectives na elke iteratie of sprint. Deze zijn niet alleen gericht op het project, maar ook op het verbeteren van het team als geheel. Het continu leren en optimaliseren blijkt van onschatbare waarde in een omgeving die voortdurend evolueert.
\newline

\section{Theorie vs praktijk}
\label{sec:theorie-vs-pratijk-sfpd}

In het algemeen omvat projectmanagement het gebruik van processen, vaardigheden, hulpmidde- len en kennis om een gepland project te voltooien en zijn doelen te bereiken. Het proces bestaat uit vijf stappen of fasen die alle projecten moeten doorlopen: initiatie, planning, uitvoering, monitoring en controle, en afsluiting.
\newline

Bij de FPD valt op dat één persoon vaak het hele project beheert, van analyse tot testen. Dit kan efficiënt zijn voor kleine projecten, maar gaat echter in tegen de principes van vele methodologie\"en, waarbij taken vaak worden verdeeld onder teamleden om de efficiëntie te verhogen en de werklast te spreiden.

Bovendien gaven beide bedrijven aan dat ze in theorie wel KPI's hebben voor elk project, maar dat vaak in tijdsnood komen, waardoor deze niet altijd worden opgevolgd. Dit benadrukt de noodzaak om een evenwicht te vinden tussen de theorie en de praktijk, waarbij de projectmanagementmethodologieën worden aangepast aan de specifieke context van een organisatie.
\newline

Ook bleek dat communicatie een belangrijke rol speelt in het succes van projecten. Bij de FPD is er een duidelijke focus op communicatie met de betrokken partijen, waarbij de projectleider de klant op de hoogte houdt van de voortgang van het project. Dit is een belangrijk aspect van projectmanagement, aangezien het de klant in staat stelt om de voortgang van het project te volgen en eventuele problemen te identificeren.

\section{Conclusie}%
\label{sec:conclusie}

In dit onderzoek hebben we een inkijkje gekregen in de diversiteit van projectmanagementmethoden binnen organisaties, waarbij elk bedrijf zijn eigen unieke benadering heeft om projecten te organiseren. Of het nu gaat om minimale analyse bij de Federale Pensioendienst, waar één persoon verantwoordelijk is voor het gehele project en een KanBan-achtige aanpak wordt gehanteerd, of de meer gestructureerde aanpak bij KBC, waar een combinatie van interne methoden en Agile-principes de huidige werkwijze bepaalt.
\newline

Bij de Federale Pensioendienst valt op dat het infrastructuurteam een relatief minimale aanpak hanteert. Met één persoon per project, belast met het opstellen van de analyse, uitvoeren van testen en het project zelf, volgen zij eerder een KanBan-geïnspireerde methode. Dit benadrukt de behoefte aan efficiëntie en flexibiliteit in kleinere projectomgevingen.
\newline

Aan de andere kant hebben we KBC, waar duidelijk verschillende richtlijnen zijn geïmplementeerd. Ze omarmen een combinatie van hun eigen methode, PLC (Project Life Cycle), en Agile principes, met de ambitie om volledig over te stappen naar het Scaled Agile Framework in de nabije toekomst. Dit getuigt van hun streven naar meer schaalbaarheid en flexibiliteit in projectmanagement, afgestemd op de steeds veranderende behoeften van hun organisatie.
\newline

Deze vergelijking benadrukt de complexiteit en de noodzaak om projectmanagementmethoden aan te passen aan de specifieke context en doelstellingen van een organisatie. Of het nu gaat om een slanke aanpak met individuele verantwoordelijkheid of een meer gestructureerde benadering met hybride methoden, het vinden van de juiste balans is essentieel voor succesvol projectmanagement.
\newline

In conclusie biedt dit onderzoek een boeiend inzicht in de diversiteit van projectmanagementpraktijken, waarbij organisaties unieke paden bewandelen om hun projectdoelen te bereiken. Elk systeem weerspiegelt niet alleen de complexiteit van de projecten zelf, maar ook de voortdurende evolutie van methodologieën om aan veranderende eisen te voldoen.
\newline

\section{Reflectie}
\label{sec:reflectie}

Het verzamelen van info verliep eerder moeizaam. Zo vonden we weinig goede papers/ artikels over ons onderwerp. Vooral omdat project management methodologi\"en meestal gemaakt en gebruikt worden voor software development. Daardoor wordt er ook meer onderzoek gedaan naar de voordelen en nadelen van verschillende stratgi\"en. Systeembeheer projecten worden minder onderzocht en dat maakte het opzoekwerk wel wat lastiger. Gelukkig hadden we heel vriendelijke en meewerkende bedrijven gevonden. Zo ondervonden wij geen weerstand bij het antwoorden van vragen. Beide bedrijven antwoorden open op onze vragen en gingen vaak veel dieper door op de vragen dan dat we verwacht hadden. Dit maakte het voor ons een zeer gemakkelijk en leuk om de interviews af te leggen. Ook wouden ze ons graag helpen met de paper en deze eens nalezen om te zien wat we er van gemaakt hebben. 

%\printbibliography[heading=bibintoc]

\end{document}
