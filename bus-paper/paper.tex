%==============================================================================
% Example hogent-article: paper in English
%==============================================================================

\documentclass[dutch]{hogent-article}

\usepackage{parskip}

\setlength{\parskip}{0pt} % Pas deze waarde aan zoals gewenst
% Specify the BibLaTeX file(s) that contain(s) your bibliography
\usepackage[backend=biber,style=apa]{biblatex}
\DeclareLanguageMapping{dutch}{dutch-apa}
\addbibresource{sources.bib}

% Course information
\studyprogramme{Paper Hoe combineer je PRINCE2 of PMBOK met agile\linebreak werken? }
\course{Business Analysis}
\assignmenttype{Paper}
\academicyear{2023-2024}
\onecolumn

% Title of the paper
\title{Hoe combineer je PRINCE2 of PMBOK met agile\linebreak werken?}

% Authors and their emails. Each should be specified separately
\author{Tibo De Wolf}
\email{tibo.dewolf@student.hogent.be}

\author{Dorus Braems}
\email{dorus.braems@student.hogent.be}

\author{Arthur Neirynck}
\email{arthur.neirynck@student.hogent.be}

\author{Sam Michiels}
\email{sam.michiels@student.hogent.be}


\newcommand{\klas}[1]{\def\@klas{#1}}
\newcommand{\groepsnummer}[1]{\def\@groepsnummer{#1}}
\newcommand{\printklas}{\@klas}
\newcommand{\printgroepsnummer}{\@groepsnummer}

\klas{G3B3}
\groepsnummer{SL_CD_06}
% Supervisor, i.e. the lecturer that will read and grade this paper
%  You can specify the role of the supervisor with an optional argument, e.g.
%   \supervisor[Advisor]{NAME}
%   \supervisor[Lecturer]{NAME}
%  Remark that there is only room for one supervisor

% Link to a Github repo where the project code is kept

% Specify the specialisation within the study programme. Choose from this
% list:
%
% - Mobile \& Enterprise development
% - AI \& Data Engineering
% - Functional \& Business Analysis
% - System \& Network Administrator
% - Mainframe Expert
% - If the subject doesn't fit, specify another (or leave the command out
%   entirely)
\specialisation{System \& Network Administrator}

% Specify some keywords that describe the paper subject
\keywords{Projectmanagement, Systeembeheer}

% Optionally, set the title/link colors to one of HOGENT's corporate brand
% colors (default for both is hogent-blue)
% See hogent-article.cls to check which colors are supported
\colorlet{title}{hogent-darkgreen}
\colorlet{links}{hogent-grey}

\begin{document}

\begin{abstract}

\end{abstract}
\bigskip
\textbf{Klas:} G3B3
\newline
\textbf{Groepnummer:} 


\newpage

\tableofcontents
\pagebreak

\section{Introductie}%
\label{sec:introductie}

In de complexe dynamiek van hedendaagse projectmanagementomgevingen staan organisaties soms voor de uitdaging om twee tegenstrijdige benaderingen te integreren: enerzijds, de behoefte van het senior management aan gestructureerde controle en voorspelbaarheid, vertegenwoordigd door gevestigde raamwerken zoals PRINCE2 of PMBOK, en anderzijds, het verlangen van wendbare teams naar flexibiliteit en betrokkenheid bij het gebruik van agile methoden. Deze spanning tussen formele projectmanagementpraktijken en de behoefte aan adaptiviteit heeft een significante impact, vooral in grotere organisaties die omvangrijke interne softwareprojecten uitvoeren.
\newline

Het vinden van een effectieve manier om deze twee schijnbaar tegenstrijdige perspectieven te verzoenen, is cruciaal voor het succes van dergelijke organisaties. Het senior management heeft vaak de voorkeur voor gestandaardiseerde raamwerken om budgetten, deadlines en inhoud te beheren, terwijl ontwikkelingsteams agile werken omarmen vanwege de flexibiliteit en de mogelijkheid om snel te reageren op veranderende vereisten. Het samenvoegen van deze twee wereldbeelden vereist niet alleen strategisch inzicht, maar ook een aanpasbare aanpak die het beste uit beide benaderingen haalt, zonder onnodige bureaucratische overhead toe te voegen.
\newline

We trachten in dit onderzoek nauwkeurig te begrijpen, hoe we gestructureerde projectmanagement frameworks zoals PRINCE2 of PMBOK effectief kunnen integreren met de flexibele werkwijzen van agile methoden in systeembeheerprojecten. We hebben deze kwestie onderzocht door middel van diepgaande analyses en interviews met senior management en uitvoerende teams in grote \linebreak organisaties.
\newline


\section{Literatuurstudie}%
\label{sec:literatuurstudie}
In een analyse door (ref) onderzoeken ze de integratie van PMBOK met Agile.
Ze doet dit door de unieke elementen van Agile te analyseren en deze in een PMBOK-benadering in te voegen.
Volgens haar onderzoek is dit mogelijk door de flexibiliteit van Agile binnen een gestructureerde PMBOK-aanpak.
(ref) komt tot de conclusie dat dit succesvol kan gebeuren als je een projectmanager hebt die een sterke kennis heeft van beide methodieken.
\newline
In de paper geschreven door (ref) wordt aangegeven dat in projecten die gebruik maken van de Agile methode Scrum er een gebrek
aan risk management technieken wordt toegepast en dat de integratie van PRINCE2 iets is dat dit kan oplossen.
Verder stelt (ref) een model voor die bestaat uit de risk management aanpak van PRINCE2 met de Scrum methodologie.
Volgens  (ref) leidt dit tot een robuster framework die meer complexe projecten tot een goed einde kan leiden.
\newline





Moderne softwareontwikkeling staat voor de uitdaging om zowel wendbaar te zijn als snel te reageren op voortdurende veranderingen. In deze context onderzoekt deze studie de hybride integratie van PRINCE2 en Scrum als een benadering die de voordelen van zowel traditionele project methodologieën als agile methodologieën benut. Door deze combinatie proberen organisaties zowel flexibiliteit als structuur te realiseren in hun softwareontwikkelingsprocessen.
\newline

Een cruciaal kenmerk van hybride softwareontwikkeling is een continue levering van werkende software, wat perfect in lijn is met de agile principes. Zoals \textcite{PoniszewskaMaranda2022} benadrukt in het Agile Manifesto, gaat de essentie van ontwikkeling, over het voortdurend toevoegen van nieuwe stukken aan werkende software over een periode van weken of maanden. Deze aanpak is flexibel en past zich aan om tegemoet te komen aan de veranderende behoeften van belanghebbenden.
\newline

De waarden en principes die zijn vastgelegd in het Agile Manifesto fungeren als richtlijnen voor de hybride aanpak. Het belang van samenwerking, klanttevredenheid en het vermogen om zich aan te passen aan veranderende vereisten \textcite{Fowler2020} weerspiegelt de kern van continue verbetering en klantbetrokkenheid in een hybride ontwikkelingsomgeving. Deze principes dienen als leidraad voor het definiëren van de werkcultuur in hybride teams.
\newline

Hoewel PRINCE2 en Scrum enige overlapping vertonen, fungeert PRINCE2 vaak als een overkoepelend projectmanagementframework, terwijl Scrum zich concentreert op specifieke agile ontwikkelingstechnieken. Zoals beschreven door het Agile Alliance (2020), omvat Scrum de Product Backlog, die wordt beheerd door de Product Owner, en het ontwikkelingsteam definieert taken om User Stories uit te voeren. Deze complementariteit in taken en verantwoordelijkheden zorgt voor een gebalanceerde en gestroomlijnde aanpak.
\newline

De synergie tussen Scrum en PRINCE2 wordt duidelijk wanneer ze worden geïntegreerd in een hybride benadering. Terwijl Scrum de ontwikkelingsteams leidt, biedt PRINCE2 een uitgebreid projectmanagementframework dat het volledige project bestrijkt. Zoals benadrukt door \textcite{PoniszewskaMaranda2022}, vullen Scrum en PRINCE2 elkaar perfect aan. Scrum levert discipline in productlevering, terwijl PRINCE2 de nodige structuur en documentatie biedt.
\newline

De integratie van PRINCE2 en Scrum in een hybride aanpak biedt organisaties een uitgebalanceerde oplossing voor complexe softwareontwikkelingsuitdagingen. Door flexibel te reageren op veranderingen, terwijl profiterend van de structuur en documentatievoordelen van PRINCE2, wordt het risico verminderd en het succespercentage van projecten verhoogd. De synergie tussen Scrum en PRINCE2 geeft organisaties de middelen om zich aan te passen aan een steeds veranderende markt en tegelijkertijd projectbeheer op hoog niveau te handhaven.
\newline

Het onderzoek van \textcite{Mousaei2018} ziet Risicobeheer als een belangrijke factor in het bepalen of een project zal slagen of niet. In het geval dat er binnen het bedrijf een correct risicobeheer is opgezet kan dit het succespercentage van projecten ver naar boven helpen. Bij vele bedrijven is het risicomanagement beperkt tot aangenomen software.
\newline 

\textcite{Mousaei2018} ziet Scrum als een van de populairste software development methodologieën en deze methodologie besteed veel tijd aan risicomanagement. In het onderzoek wordt een nieuw model voor risicomanagement voorgesteld dat het Scrum-framework combineert met het Prince2-framework voor projectmanagement. Het model is ontwikkeld met medewerking van 52 Agile-experts uit zes verschillende landen.
\newline

 De hoofddoelen van dit model zijn het verbeteren van de dekking en het juiste mechanisme voor risicobeheer bij softwareprojecten, het verhogen van de slagingskans van het project, een goede \linebreak schatting  van de benodigde tijd, het verbeteren van de productkwaliteit en het verbeteren van \linebreak kwaliteitsparameters zoals bruikbaarheid, flexibiliteit, efficiëntie en betrouwbaarheid.

\section{Interviews}
\label{sec:interviews}
Voor ons onderzoek hebben we twee interviews afgelegd om een idee te krijgen
in hoe bedrijven de combinatie van agile met PRINCE2/PMBOK aanpakken.
Het eerste interview vond plaats via MS Teams bij Inetum-realdolmen
, waar we een idee kregen van hoe zij deze combinatie teweegbrengen. Het tweede interview vond fysiek plaats
bij Cegeka waar we een andere visie kregen op het probleem. Deze interviews stellen ons in staat een beeld te schetsen van hoe bedrijven strategieen toepassen om de methodes te combineren

\subsection{Gestelde vragen}%
\label{ssec:gestelde-vragen}

Tijdens deze interviews hebben we de volgende vragen gesteld:

\subsubsection{Huidig Project Management Framework}
\label{ssec:Huidig Project Management Framework}

\begin{itemize}
    \item Hoe zou je het huidige project management framework binnen het bedrijf beschrijven? Maakt het gebruik van PRINCE2/PMBOK, agile, of een combinatie daarvan?
\end{itemize}

\subsubsection{Mogelijkheden van een Hybride PRINCE2/PMBOK en Agile Aanpak}
\label{ssec:Mogelijkheden en uitdagingen van een Hybride PRINCE2/PMBOK en Agile Aanpak}

\begin{itemize}
    \item Hoewel Agile flexibeler is, kan het een aantal  problemen met zich meebrengen zoals: scope creep, onrealistische verwachtingen, gebrek aan samenwerking en gebrek aan communicatie. Denkt u dat een combinatie van Prince2/PMBOK en agile deze problemen kan verhelpen?
\end{itemize}

\subsubsection{Integratie PRINCE2 Agile - Balanceren tussen Productgerichtheid en Flexibiliteit}
\label{ssec:Integratie PRINCE2 Agile - Balanceren tussen Productgerichtheid en Flexibiliteit}

\begin{itemize}
    \item PRINCE2 helpt om te bepalen welke producten nodig zijn om aan de business case te voldoen, terwijl Agile-projecten zich meestal richten op het maken van één product en dat goed doen. Hoe brengt PRINCE2 agile deze 2 perspectieven samen en zorgt het ervoor dat er geen essentiële producten over het hoofd worden gezien, terwijl het tegelijkertijd de flexibiliteit van Agile benut om de benodigde producten te creëren?
\end{itemize}

\subsubsection{Uitdagingen bij Integratie van PRINCE2 en Agile in de Organisatie}
\label{ssec:Uitdagingen en Overwinningen bij Integratie van PRINCE2 en Agile in de Organisatie}

\begin{itemize}
    \item Het mengen van Agile met PRINCE2 zou in eerste instantie gezien kunnen worden als het toevoegen van zware processen aan Agile. Zijn er uitdagingen of scepticisme geweest van PRINCE2 of Agile beoefenaars binnen uw organisatie, en hoe hebt u deze zorgen aangepakt?
\end{itemize}

\subsubsection{Uitdagingen Agile met PRINCE2 - Budget en Timing}
\label{ssec:Uitdagingen Agile met PRINCE2 - Budget en Timing}

\begin{itemize}
    \item Wat zijn de uitdagingen of problemen die Agile met PRINCE2 ervaart in verband met budget en timing?
\end{itemize}

\subsubsection{Tools voor Integratie PRINCE2/PMBOK en Agile}
\label{ssec:geslaagde-en-niet-geslaagde-projecten}

\begin{itemize}
    \item Welke tools worden zoal gebruikt om PRINCE2/PMBOK en agile met elkaar te integreren ?
\end{itemize}

\subsection{Cegeka}
\label{ssec:Inetum-Cegeka}

Het project management framework binnen Cegeka bestaat vooral uit een agile aanpak. Bij de kleinere projecten zal er voor een scrum-methodiek gekozen worden.
Indien de projecten groter worden komt er wel een combinatie van Agile met Prince2 elementen. Dit omvat een uitgebreide analysefase waarin aspecten zoals risico,budget en timing bekeken worden en dit enkele maanden voor het project van start gaat. Bovendien implementeren ze quality gates, op die manier wordt het project zodanig opgevolgd dat de criteria worden bereikt tijdens elke fase van het project.
\newline

Om problemen in agile werken zoals scope creep, onrealistische verwachtingen tegen tegaan. Moet er een sterke communicatie zijn tussen de productowner en de projectmanager, de productowner moet iemand zijn die beslissingen kan maken. Volgens hun is dit noodzakelijk om niet in problemen te komen met scope creep want anders blijf je functies implementeren waarvoor je eigenlijk niet de tijd hebt ingepland en het budget niet hebt. 
\newline

Prince2/PMBOK elementen toevoegen aan een agile aanpak is voor hun geen uitdaging. De mensen die bij hun komen werken weten wat hun stijl van werken is en hebben hier geen bezwaar tegen.
Als een projectleider zijn processen niet opgevolgd komt die toch in de problemen. Zo vinden zij dat er geen tegenstrijding is tussen agile en PRINCE2 elementen.
\newline
Bij een combinatie van Agile en Prince2 moet er gebruik gemaakt worden van stuurgroepen. Dit is iets dat je niet
terug ziet in traditionele Agile projecten. Het is zeer belangrijk voor hun dat deze groepen een sterke connectie hebben met de klant.
Want dit is waar er beslissingen moeten worden genomen op basis van timing en budget, een voordbeeld die ze gaven was stel dat we 
plots nood hebben aan twee nieuwe developers om ons project tijdig klaar te krijgen, dan moeten deze stuurgroepen met
de klant samenzitten om budget en timing te gaan bespreken want deze developers werken natuurlijk niet gratis.
\newline
Op vlak van tools gebruiken ze vooral JIRA en Azure DevOps ook omdat ze willen gebruik maken van een ticketsysteem.
Jira wordt vaak gebruikt in agile projecten door de flexibiliteit en aanpasbaarheid van de tool dit maakt het ook geschikt voor een combinatie van Agile en PRINCE2/PMBOK.
Azure DevOps bied ook een breed scala aan functionaliteiten voor projectbeheer.
Hun financiele dienst zal wel eerder gebruik maken van een Excel, maar de projectleiders gebruiken voornamelijk deze twee tools.


\subsection{Inetum-realdolmen}
\label{ssec:interview-Inetum-realdolmen}
Projecten bij Inetum-realdolmen worden vooral opgesplitst. Voor sommige projecten gebruiken ze PRINCE2, terwijl ze het beter vinden om voor andere projecten Agile te gebruiken. Ze benadrukten wel dat als je kiest voor één projectmanagement systeem, je daardoor geen ander systeem uitsluit, zo zijn er zaken die bij Agile werken, die ook bij een traditionele PRINCE2/PMBOK aanpak relevant zijn. 
\newline

Ook wilden we weten aangezien Agile een aantal problemen met zich mee kan brengen zoals scope creep, onrealistische verwachtingen, gebrek aan samenwerking en gebrek aan communicatie, of een combinatie met PRINCE2/PMBOK deze problemen zouden kunnen verhelpen. Om deze vraag te beantwoorden werd ons een voorbeeld gegeven. Dit voorbeeld was dat PRINCE2/PMBOK bijvoorbeeld meer geschikt zijn als je een brug gaat bouwen en je weet exact hoe die er moet uitzien en je hebt daar exacte acties bij. Hierbij is er dus geen sprake van scope creep. 
\newline

Agile gebruiken wij meer als er waarde opgeleverd moet worden  en wanneer je vooral met de klant nog in afstemming bent hoe het er juist moet uitzien. Bijvoorbeeld je gaat een website bouwen waarbij de klant wel een idee heeft van hoe die er moet uitzien maar dit is niet ‘carved in stone’, of in andere woorden niet exact . Je gaat dus met de klant gaan afstemmen wat je gebouwd hebt en of dit volgens de verwachtingen is en hier kan de klant dan feedback op gegeven, voor eventuele veranderingen. Dit is het fundamentele verschil tussen PRINCE2/PMBOK en Agile qua Scope creep. 
\newline

Hierna werd ons de persoonlijke visie verteld over projectmatig werken en de evolutie met Agile werken. Bij Agile werken ligt de nadruk heel hard op het individu en de individuele benadering. Ook ligt de nadruk op de samenwerking met de klant. Er is een heel  grote nadruk op samenwerking in team, zodat developers elkaar kunnen steunen. Terwijl bij PRINCE2/PMBOK heeft iedereen zijn eigen taak dus is er minder wisselwerking. Maar samenwerken is in beide gevallen wel zeer belangrijk. Als je dan gaat kijken naar het Agile manifesto dat dateert van 2001, daarvoor was er PRINCE2/PMBOK waar je alles zeer strikt en afgemeten ging gaan volgen, bijna procedure matig en dan bekijk je de evolutie dan zie je dat ook het menselijke aspect belangrijk is. 
\newline

Het Agile manifest is er gekomen in de jaren 2000, dat is ook de periode dat Lencioni zijn boek rond Five Dysfunctions of a Team heeft geschreven. Mijn persoonlijke mening is, dat het menselijke aspect zinvol en nodig is. 1 plus 1 is drie zeggen we bij Inetum-realdolmen ook nogal eens. Dus die samenwerking kan heel veel opleveren en dit is belangrijk bij beide types projecten. Zo zie je bij PRINCE2/PMBOK dat je daar die teamsamenstelling van Lencioni ook bij neemt en dan ben jij eigenlijk die 2 samen een stukje Agile aan het doen. 
\newline

Vervolgens wilden we weten hoe PRINCE2/PMBOK en Agile de 2 perspectieven van flexibiliteit en geen producten over het hoofd zien, samenbrengen. Er werd ons verteld dat we bij deze vraag uitgingen van een aantal veronderstellingen. We gingen ervan uit dat PRINCE2/PMBOK een aantal producten oplevert op basis van de business case en Agile dit eerder gaat opdelen in kleinere stukken, in sprints. 
\newline

Er werd ons verteld dat hier het principe van een ‘work in progress’ zit. Dit betekent dat als je te veel tegelijk door de projectmachine haalt dat dit eigenlijk niet oplevert.Als er te veel dingen tegelijk spelen dan ga je veel minder snel en minder goed iets kunnen opleveren omdat je met alles tegelijk bezig bent. Het is een heel belangrijk principe binnen agile dat je de prioriteiten stelt voor kleine behapbare brokken waar iedereen zich toe gecommit heeft. 
\newline

Hun mening was dat bij Agile het belangrijkste is om in contact te  staan met je klant. Bij PRINCE2 en PMBOK ook, maar daar heb op voorhand meer tijd besteed om te bepalen wat de concrete onderdelen zijn en de producten die moeten worden opgeleverd. Hier hou je je meer aan vast. Dit was volgens hen het grote verschil tussen PRINCE2/PMBOK en Agile. 
\newline

Er werd ons verteld dat bij Agile er een aantal zaken vast zijn, zoals de kosten en resources en doorlooptijd maar wat je eigenlijk gaat bouwen is flexibel. Bij PRINCE2/PMBOK is het andersom, wat je gaat bouwen staat vast, maar de tijd en kosten en resources, daar kan mee gespeeld worden. Bij projecten waar je sneller klaar wil zijn, gebruiken ze dan meer resources of halen ze er een expert bij.
\newline

Ze waren dus van mening dat als ze deze aspecten binnen deze ideologieën goed managen, beiden een goed resultaat opleveren. 
\newline

Vervolgens wilden we weten of het toevoegen van PRINCE2/PMBOK componenten aan een Agile werkwijze, niet gezien kon worden als het toevoegen van nodeloos zware processen aan een flexibele methode.  Hierbij werd ons een concreet voorbeeld gegeven. 
\newline

Er werd ons gesteld dat als je bijvoorbeeld een vliegtuig gaat bouwen, je niet soft moet beginnen gaan doen en dat dan de documentatie echt wel in orde moet zijn. Bij dit soort projecten ga je dan meer op aspecten zoals planning gaan focussen en dan vinden zei de PRINCE2 aanpak beter. 
\newline

Daarnaast hebben ze ervaren dat bij projecten waar er 2 leveranciers waren, waarbij 1 Agile werkte, dat mensen vooral energie krijgen van samen een eindproduct te maken. Ook door te weten wat de klant exact wil gaan ze die functionele/business analyse al schrijven. Dus dan zijn ze over heel de lijn betrokken met de klant. Hierbij gaven ze het voorbeeld van een auto maken. Vroeger was dat bandwerk en iedereen deed zijn eigen deeltje, maar nu zeg je we gaan allemaal samen zorgen dat die auto gebouwd wordt. Ze vinden het meer motiverend en zingevend of ‘meer van deze tijd’ om op die manier te werken. Met dit voorbeeld bedoelden ze dus samenwerken op een Agile manier naar een eindproduct. 
\newline

Deze methode staat dan tegenover de andere leverancier die het op de PRINCE2/PMBOK manier deed. Als ze een developer nodig hadden, dan was dat bijvoorbeeld een Adobe developer en niets anders, dat was zeer rechtlijnig, met de filosofie van: dat is de vraag en dat heb ik exact nodig en dan gewoon uitvoeren. Hierbij krijgen we ook wel rewards want die hebben geen problemen, geen vragen en die moeten ook minder afstemmen. Die krijgen iets binnen, duiden dan alle vinkjes aan, en klaar. Hierbij vroegen we of het team hierbij problemen had: “Vinden wij de mix tussen de 2 dan haalbaar ? Het gaf wel problemen in het team, vooral de manier van werken dan. Omdat de ene van de andere niet begreep waarom bijvoorbeeld de Agile mensen mee in de klant vergadering moesten zitten. Zodat ze de klant goed begrepen en dan zelf bijdragen aan de oplossing. Dus het kan wel problemen geven” Toch geloven zijn bij Inetum-realdolmen er sterk in, afhankelijk van het type product of het type project dat je moet opleveren, er een juiste methodologie bij past.
\newline

Hierbij wouden we nog wat meer duidelijkheid dus vroegen we of ze dachten dat het een beter idee was om, in plaats van 2 methodologieën te gaan mixen, om ze te gebruiken in de juiste context van het project. Hierbij maakte ze de koppeling terug naar het begin van het interview over projectmatig werken met PRINCE2/PMBOK en Lencioni. 
\newline

Wat ze vooral belangrijk vinden is hoe het team gaat samenwerken en hoe ze zorgen dat conflicten op een goede manier worden aangegaan, dat mensen zelf hun eigen inbreng hebben. Bij de manier waarop ze Agile werken, zit dat mee in de ceremonie, in de retro’s, in de demo’s. 
\newline

Ze vinden dat je allebei de methodes nodig hebt. Tenzij je echt routine werk gaat doen, bij zaken die je al 100 keer gedaan hebt. Zaken waar je niet meer over hoeft af te stemmen. Waar iedereen zijn taak weet en wat hij moet uitvoeren.
\newline

Dus ze zoeken de methode die past bij de producten die ze maken en bij het soort mensen. Er zijn mensen die graag routinematig werken en er zijn mensen die graag zelf creatief zijn. En ze proberen bijvoorbeeld developers, die graag creatief zijn, vooral Agile te laten werken omdat die manier het best bij hen past. Er werd verteld dat Agile ook het meest in development wordt gebruikt, vooral omdat het eindproduct nog samen met de klant wordt uitgewerkt. 
\newline

Tijdens de volgende vraag gaven we eerst een schetsing: Als je PRINCE2/PMBOK gebruikt zet je het eindresultaat eigenlijk al een beetje vast en het budget dat daarvoor wordt gebruikt, terwijl als je Agile werkt, dan kan dat aangepast worden. Dus ons lijkt het moeilijk om die 2 methodologieën samen te voegen ? Omdat die methodes contradicterend lijken en haaks staan op elkaar. Hierop zeiden ze dat dit tegelijk waar en onwaar was, in die zin dat in een klassiek PRINCE2/PMBOK project ze ook de mogelijkheid hebben om veranderingen toe te laten. Dus bijvoorbeeld bij scope creep, als ze zien dat ze aan het afwijken zijn van datgene dat we gebudgetteerd hadden en waar ze mensen voor hebben of de klant wil plots iets anders waar ze meer budget en mensen voor nodig hebben. Dan nemen ze dat goed op en  laten ze daar een change traject op los. Om de scope creep te vermijden budgetteren ze dan opnieuw op de PRINCE2/PMBOK manier. 
\newline

Ze vertelden dat beide methodologieën waakzaamheid nodig hebben, ‘common sense’ en ook veel communicatie met de klant. Want zelfs bij een Agile verhaal, waarbij ze de scope nog aan het bouwen zijn, wil iedere klant toch dat het binnen het budget en binnen de tijd opgeleverd wordt. Dus de principes vinden ze theoretisch leuk maar hun klanten blijven business gewijs hetzelfde. Dus voor beide methodes denken ze dat je problemen vermijdt door open communicatie, goed budgetbeheer, goed tijdsbeheer.
\newline

Om het interview af te sluiten wilde we ook nog weten welke tools er worden gebruikt binnen Inetum-realdolmen om PRINCE2/PMBOK en Agile met elkaar te integreren. 
\newline

Hierover zeiden ze dat Lencioni wel heel belangrijk is. Het is in hun ervaring zeer zinvol om voor mensen die in het agile gestart zijn en daarin gegroeid zijn om die business sense te behouden. Wat ze soms zien gebeuren is, dat sommige mensen heel vrijblijvend zijn en dat ze echt op basis van heel weinig requirements aan de slag gaan. Ze werken dan via het principe: “Hoe groter de chaos hoe beter want dan zijn we agile bezig”, terwijl ze vinden dat het toch in ieders belang is om dingen goed af te stemmen, zoals bijvoorbeeld ‘the definition of done’ of de acceptatiecriteria. Ze vinden het menselijk aspect en het business aspect even belangrijk blijven.


\section{Conclusie}%
\label{sec:conclusie}
De integratie van traditionele projectmanagementraamwerken zoals PRINCE2 of PMBOK met agile methodologieën vormt een complexe uitdaging in hedendaagse projectmanagementomgevingen. De behoefte aan gestructureerde controle en voorspelbaarheid vanuit het senior management, vertegenwoordigd door gevestigde raamwerken, botst vaak met de flexibiliteit en betrokkenheid die gewenst wordt door agile teams. Deze spanning tussen formele projectmanagementpraktijken en de vraag naar aanpassingsvermogen is vooral voelbaar in grotere organisaties die betrokken zijn bij omvangrijke interne softwareprojecten.
\newline

Het doel van het onderzoek in deze studie was om uitgebreid te begrijpen hoe gestructureerde projectmanagement frameworks zoals PRINCE2 of PMBOK effectief kunnen worden geïntegreerd met de flexibele praktijken van agile methoden in systeembeheerprojecten. Door middel van diepgaande analyses en interviews met senior management en uitvoerende teams in grote organisaties, probeerde het onderzoek strategieën bloot te leggen voor het bereiken van een balans tussen deze schijnbaar tegenstrijdige perspectieven.
\newline

Uit de literatuurstudie bleek dat een hybride integratie van PRINCE2 en Scrum een oplossing zou kunnen bieden die de voordelen van zowel traditionele als agile methodologieën benut. Deze hybride aanpak stelt organisaties in staat om zowel flexibiliteit als structuur aan te brengen in hun softwareontwikkelingsprocessen. 
\newline

Onderzoeken benadrukte het belang van continue levering van werkende software, in lijn met agile principes, en benadrukte de synergie tussen Scrum en PRINCE2 in het bieden van een evenwichtige en gestroomlijnde aanpak.
\newline

Onze interviews met Cegeka en Inetum-realdolmen gaven ons inzicht in de praktijk. Cegeka, bijvoorbeeld, beschreef hun projectmanagementraamwerk als overwegend Agile, met elementen van Scrum voor kleinere projecten en een hybride aanpak voor grotere projecten. Ze benadrukten het belang van sterke communicatie tussen de producteigenaar en de projectmanager om uitdagingen zoals scope creep in agile projecten aan te pakken.
\newline

Inetum-realdolmen bespraken hun gebruik van zowel PRINCE2/PMBOK als Agile, afhankelijk van de aard van het project. Ze erkenden dat de integratie van deze methodologieën een doordachte aanpak vereist, rekening houdend met factoren zoals de samenwerking binnen het team, individuele voorkeuren en het soort project. Ze benadrukten het belang van het gebruik van de juiste methodologie in de juiste context en het beheren van potentiële conflicten binnen het team.
\newline



\section{Reflectie}
\label{sec:reflectie}
<<<<<<< HEAD

\printbibliography[heading=bibintoc]
=======
Het verzamelen van informatie verliep vrij vlot er waren genoeg papers die het over het balanceren van
een Agile aanpak met een PRINCE2/PMBOK hadden, dit omdat het een combinatie is die vaak terugkomt in
projectmanagementframeworks voor software development projecten. Het vinden van bedrijven die ons te woord wouden staan ging iets moeizamer.
Maar uiteindelijk hebben we toch twee projectleiders gevonden die ons wouden helpen. De mensen die de
interviews aflegden waren zeer vriendelijk en gepassioneerd in hun vak, beide bedrijven antwoorden zeer open op onze vragen en gingen er ook vrij diep op in.
Het antwoord op enkele vragen week wel wat af van wat er gesteld werd.

%\printbibliography[heading=bibintoc]
>>>>>>> d09ecc0 (Interview,Literatuur,Reflectie)

\end{document}
